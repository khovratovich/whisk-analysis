\documentclass{article}
\usepackage[utf8]{inputenc}

\usepackage{amssymb,amsmath,amsfonts,longtable,amsthm,
booktabs, comment,array, ifpdf, tabularx,url,color,array,colortbl,hyperref,graphicx, physics,siunitx}
\usepackage[lambda,advantage,adversary, landau, probability, notions, logic, ff, mm, events, complexity, oracles, asymptotics, keys, primitives, operators, sets]{cryptocode}
\usepackage[hmargin=3cm,vmargin=3cm]{geometry}


\newcommand{\dmitry}[1]{ {\color{cyan}{ Dmitry: #1}} }
\newtheorem{definition}{Definition}
\newtheorem{lemma}{Lemma}
\newtheorem{corollary}{Corollary}
\newtheorem{theorem}{Theorem}
\newtheorem{proposition}{Proposition}
\newtheorem{remark}{Remark}
\newtheorem{assumption}{Assumption}

\newcommand{\Gen}{{\mathsf{Gen}}}
\newcommand{\Eval}{{\mathsf{Eval}}}
\newcommand{\Verify}{{\mathsf{Verify}}}
\newcommand{\pubparam}{{\mathsf{pp}}}
\newcommand{\id}{{\mathsf{id}}}
\newcommand{\diffparam}{\Delta}
\newcommand{\timestamp}{t}
\newcommand{\domain}{{\mathcal{X}}}
\newcommand{\range}{{\mathcal{Y}}}
\newcommand{\resource}{{\mathcal{R}}}
\newcommand{\usample}{\xleftarrow{\$}}

\usepackage[normalem]{ulem}
\usepackage[capitalise]{cleveref}
\hypersetup{
    colorlinks=true,
    linkcolor=blue,
    filecolor=magenta,      
    urlcolor=blue,
    pdfpagemode=FullScreen,
}


\title{Privacy Analysis of Whisk}
\author{ Dmitry Khovratovich}


\begin{document}

\maketitle

\section{Notation}
\begin{itemize}
    \item $N$ is the number of trackers;
    \item $K$ is the stir size;
    \item $s$ is the number of rounds (with $N/K$ stirs in each).
\end{itemize}

\section{Feistel stir selection rule}

We represent trackers as a $K\times K$ matrix $M$. Then define $F(x,y)$ as
$$
F(x,y) = (y, x+y^3 \bmod{K}).
$$

Finally, we  define  that $i$-th proposer of $r$-th round selects for stirring the $i$-th row of matrix $F^k(M)$.

We say that the tracker selection rule is \emph{uniform} if all trackers of a single stir will be processed by distinct proposers in the next and in the after-next rounds.


\begin{proposition}
The Feistel rule is uniform for $K=128$.
\end{proposition}
\begin{proof}
The proof is easy: for $y_1\neq y_2$ we have
\begin{align*}
F(x,y_1) &= (y_1,x+y_1^3);& F^2(x,y_1) &= (x+y_1^3,y_1+(x+y_1^3)^3);\\ F(x,y_2) &= (y_2,x+y_2^3);& F^2(x,y_2) &= (x+y_2^3,y_2+(x+y_2^3)^3)
\end{align*}
where  $x+y_1^3\neq x+y_2^3$ as $y\mapsto y^3$ is bijective modulo 128.
\end{proof}

The uniformity of the Feistel rule implies that 1- and 0-touchers of each round are uniformly spreaded to stirs of the next rounds, which can be formulated as follows.

\begin{proposition}\label{prop:f2}
Let $\mathcal{S}$ be any subset of  stirs with a uniform rule such that in round $r$ fraction $\alpha_r$ of stirs are in $\mathcal{S}$. Then the fraction of 1-touchers after round $k$ is $$
F_1(k)=\prod_{r\leq k} (1-\alpha_r).
$$
\end{proposition}

\section{Privacy analysis}




Here we prove bounds on the censorship costs. First we outline the adversarial strategy that we presume optimal:
\begin{itemize}
    \item Adversary knows the stirs of fraction $\beta$ proposers.
    \item There are additionally $\gamma$ proposers that go offline every day, and they are known to the adversary.
    \item During the day, the adversary kills all 0-touchers as they are the cheapest.
    \item He orders the remaining $(1-\beta-\gamma)$ trackers by the anonymity set size. 
    \item Adversary shuts 1-touchers starting from the least anonymous ones.
\end{itemize}

Now we utilize the property of the Feistel rule.


\begin{proposition}
Let fraction $\alpha$ of proposers be honest and alive. Then the total number of 0-touchers before final filtering is at most $$
W(\alpha) =\frac{N+20\sqrt{N}}{\alpha K}
$$

\end{proposition}
\begin{proof} Let us find how many trackers can evade the fraction $\alpha$ of stirs and thus become 0-touchers.

For some $W$ bigger than $K$ let us select randomly $W$ trackers  and a single stir. The probability for all the trackers to miss the stir is $p=(1-\frac{K}{N})^W\approx e^{-\frac{WK}{N}}$. Consider  stirs $\mathcal{S}_j=\{S[1,j],S[2,j],\ldots,S[s,j]\}$ i.e. those  that are $j$-th in their round. The number $\nu_j$ of stirs in $S$ not touching any of those $W$ trackers is a random variable, which is the sum of $s$ independent Bernoulli variables with mean $p$, and so has Binomial distribution with parameters $(s,p)$. Note that random variables $\nu_j$  have negative covariance, so the total number $\nu$ of stirs not touching any of $W$ trackers  can be upper bounded by the variable with distribution $Bin(N/K\cdot s= N/2,p)$. The latter distribution can be approximated by normal one with parameters $(\mu=Np/2,\sigma^2=Np(1-p)/2)$.  With probability $e^{-80}$ we have that  the value of $\nu$ is at most $$
X(W)=\mu+12\sigma = Ne^{-\frac{WK}{N}}/2+12e^{-\frac{WK}{2N}}\sqrt{N/2}\approx \frac{N^2/2+10N\sqrt{N}}{WK}
$$. We thus assume that at most $2^{-128}$ such sets of $W$ trackers miss more than $X$ stirs, and they would be infeasible to find. Thus the total fraction of stirs that can be evaded is $X/(N/2) = \frac{N+20\sqrt{N}}{WK}$, and so is  the maximum fraction of honest proposers that miss $W$ trackers.
\end{proof}
For $N = K^2 = 2^{16}$ we have $W(
\alpha) = \frac{148}{\alpha}$.

Now note the following
\begin{itemize}
    \item The anonymity set of each tracker increases with each honest stir it undergoes. 
    \item Denote the fraction of 1-touchers after $r$ rounds by $F_1(r)$. Since honest and online proposers are uniformly distributed over rounds thanks to the final filtering, and as the Feistel rule is uniform (see \cref{prop:f2}), we have
    $$
    F_1(r)\approx 1-(1-\alpha)^r.
    $$
    This approximation is good enough for the anonymity set estimate.
    \item Thanks to the uniformity of dispersion, we have each stir of round $r+1$ taking the same fraction of 1- and 0-touchers. Therefore out of $(1-\beta)K$ benign trackers we have $F_1(r)(1-\beta)K$ 1-touchers each with anonymity set at least $(1-\beta)K$, and the anonymity set of 1-touchers last touched at round $r+1$ is at least
    $$
    A_1(r+1) = (1-(1-\alpha)^r)(1-\beta)^2K^2 
    $$
    \item For 1-toucher to be last touched in round $r$, it must undergo no honest stirs after that, which happens with probability $(1-\alpha)^{s-r}$. Thus the fraction of trackers last touched in round $r$ is
    $F_2(r)=\alpha(1-\alpha)^{s-r}$.
\end{itemize}


\begin{proposition}
Suppose that the attacker shuts in day 2 down all nodes that were last touched in round $r$ or earlier of day 1. Then the cost is
\begin{equation}
   C(r)\geq \sum_{k=1}^r A_1(k)F_2(k)N =\sum_{k=1}^r (1-(1-\alpha)^{k-1})\alpha(1-\alpha)^{s-k}(1-\beta)^2K^2N.
\end{equation}

\end{proposition}

\begin{theorem} Let the adversary
\begin{itemize}
    \item control $\beta N$ proposers in period $i$ (i.e. she knows the shuffles of those).
    \item be able to shut down arbitrary $\delta N$ proposers in period $i$. 
\end{itemize} 
Let the fraction $\gamma$ of honest proposers of each day go offline. 
Then the fraction of honest proposers of each day is $\alpha = 1- \beta-\gamma-\delta$ and the cost of the attack is $C(r_0)$
 where $$
r_0= \log_{1-\alpha}(1-\delta + \frac{W(\alpha)}{2N})
$$

\end{theorem}
\begin{proof} In day 1 the fraction of honest and alive proposers who stir is $\alpha = 1- \beta-\gamma-\delta$. By Lemma the maximum number of trackers that become 0-touchers is $W(\alpha)$, of which 1/2 goes to Day 2. Therefore, out of $\delta N$ attacked trackers in Day 2, at least $\delta N - W(\alpha)/2$ are 1-touchers. The attacker then shuts down the 1-touchers that were touched at round $r_0$ at latest such that
$$
\sum_{k\leq r_0}F_2(k)\geq \delta - \frac{W(\alpha)}{2N}
$$
which is equivalent to \begin{align}
\sum_{k\leq r_0}F_2(k)\geq \delta - \frac{W(\alpha)}{2N}\;\Leftrightarrow\\
\sum_{k\leq r_0}\alpha(1-\alpha)^{s-k}\geq \delta - \frac{W(\alpha)}{2N}\;\Leftrightarrow\\
\alpha(1-\alpha)^{s-r_0}\frac{(1-\alpha)^{r_0}-1}{-\alpha}\geq \delta - \frac{W(\alpha)}{2N}\;\Leftrightarrow\\
(1-\alpha)^{s-r_0}-(1-\alpha)^s\geq \delta - \frac{W(\alpha)}{2N}\;\Leftrightarrow\\
r_0 = s-\log_{1-\alpha}\left(  \delta - \frac{W(\alpha)}{2N}+(1-\alpha)^s\right)
\end{align}
The cost is then $C(r_0)$.
\end{proof}

\end{document}