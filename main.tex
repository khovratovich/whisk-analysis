\documentclass{article}
\usepackage[utf8]{inputenc}

\usepackage{amssymb,amsmath,amsfonts,longtable,amsthm,
booktabs, comment,array, ifpdf, tabularx,url,color,array,colortbl,hyperref,graphicx, physics,siunitx}
\usepackage[lambda,advantage,adversary, landau, probability, notions, logic, ff, mm, events, complexity, oracles, asymptotics, keys, primitives, operators, sets]{cryptocode}
\usepackage[hmargin=3cm,vmargin=3cm]{geometry}


\newcommand{\dmitry}[1]{ {\color{cyan}{ Dmitry: #1}} }
\newcommand{\pratyush}[1]{ {\color{blue}{ Pratyush: #1}} }
\newcommand{\mary}[1]{ {\color{magenta}{ Mary:  #1}} }
\newtheorem{definition}{Definition}
\newtheorem{lemma}{Lemma}
\newtheorem{corollary}{Corollary}
\newtheorem{theorem}{Theorem}
\newtheorem{proposition}{Proposition}
\newtheorem{remark}{Remark}
\newtheorem{assumption}{Assumption}

\newcommand{\Gen}{{\mathsf{Gen}}}
\newcommand{\Eval}{{\mathsf{Eval}}}
\newcommand{\Verify}{{\mathsf{Verify}}}
\newcommand{\pubparam}{{\mathsf{pp}}}
\newcommand{\id}{{\mathsf{id}}}
\newcommand{\diffparam}{\Delta}
\newcommand{\timestamp}{t}
\newcommand{\domain}{{\mathcal{X}}}
\newcommand{\range}{{\mathcal{Y}}}
\newcommand{\resource}{{\mathcal{R}}}
\newcommand{\usample}{\xleftarrow{\$}}

\usepackage[normalem]{ulem}
\usepackage[capitalise]{cleveref}
\hypersetup{
    colorlinks=true,
    linkcolor=blue,
    filecolor=magenta,      
    urlcolor=blue,
    pdfpagemode=FullScreen,
}


\title{Princes and Paupers:\\
Design and Analysis of a Multiparty Shuffle}
\author{ Dmitry Khovratovich \and Mary Maller}


\begin{document}

\maketitle

\section{ Functionality Requirements}

The timeline is divided into \emph{periods}. The total of $N$ blocks are added to the chain during each period. The $i$-th block is added as follows:
\begin{enumerate}
    \item At eligible timeslot $\mathcal{T}_i$ a proposer $P_{j_i}$ broadcasts block $B_i$ along with the eligibility proof $\pi_i'$.
    \item The timeslot $\mathcal{T}_i$ is defined on another level of the Ethereum protocol which is beyond the scope of this report.
    \item The eligibility proof $\pi_i'$ binds together:
    \begin{itemize}
        \item The proof $\pi_i$ that $P_{j_i}$ can propose at $\mathcal{T}_i$.
        \item The block $B_i$.
    \end{itemize}
    This can be achieved by $hash(B_i)$ being part of the context of the proof $\pi_i$.
\end{enumerate}

There are $L$ validators competing for a proposer place in each period. For period $i$ they are selected as follows:
\begin{enumerate}
    \item During period $(i-2)$ each candidate submits  a commitment $C_j$ to some private value $c_j$. There is no private channel so all parties know who submitted what.
    \item Before period $(i-1)$ starts the total of $N'$ commitments are selected using some initial filtering.
    \item During period $(i-1)$ each block proposer mixes (\emph{shuffles}) a certain subset of commitments in a way that if $C_j$ is shuffled then the owner $C_j$ still can find it in the output and prove the knowledge of $c_j$ but a non-owner should not be able to. Each block contains the output set $\{\widehat{C}_j\}$ of commitments.
    \item In the end of period $(i-1)$ the $N'$ (shuffled) commitments are additionally filtered down to $N$ commitments. The owners of the latter are  proposers for period $i$.
\end{enumerate}

\subsection{Network model}

Non-adversarial nodes operate as follows:
\begin{itemize}
    \item They are either online or offline (faulty) during the entire period;
    \item Their network address is available to the adversary.
    \item They do not disclose they are eligible proposers till the very moment of broadcast.
\end{itemize}

\section{Security requirements}

\subsection{Adversary model}

An adversary is a network attacker who attempts to censor the chain construction in two ways:
\begin{itemize}
    \item Making himself many of block proposers in each period;
    \item Eclipsing/DoSing the proposers he does not control.
\end{itemize}

\section{External components}


\subsection{RANDAO}

Here we explain how RANDAO works.

\subsection{VDF}

Here we explain how a future VDF would work.

\section{Design}

We set $d=2$ i.e. the initial filtering selects $2N$ commitments for shuffling, and the final filtering selects $N$ out of $2N$.

\subsection{Eligibility proofs}

Here we explain how SSLE proofs work

\subsection{Stir}

Here we explain how each proposer shuffles their working set. The size of the working set is $K$. For convenience we set $K$ to be a factor of $N$.

\subsection{Dispersion}

Here we explain how working sets spread and mix among shufflers. It is convenient to view the set of commitments as a matrix $C[i][j]$ of dimensions $N/K \times K$.

The design is as follows:
\begin{itemize}
    \item The period is split into $K$ \emph{rounds}, where each round covers $N/K$ block proposals.
    \item The working sets for each round are $N/K$ rows $C[i][*]$ of $K$ commitments each.
    \item In the end of each round the commitments undergo the dispersion step: $C[i][j]\leftrightarrow C[(i+j) \pmod{N/K}][j]$.
\end{itemize}


\subsection{VDF upgrade}

Here we explain how we plan to replace RANDAO with VDF.

\section{Analysis}

\subsection{Security of eligibility proof}

Here we show that the eligibility proof is unforgeable.

\subsection{Cost analysis of censorship}


Here we prove bounds on the censorship costs. First we outline the adversarial strategy that we presume optimal:
\begin{itemize}
    \item Adversary knows the stirs of fraction $\beta$ proposers.
    \item There are additionally $\gamma$ proposers that go offline every day, and they are known to the adversary.
    \item During the day, the adversary kills all 0-touchers as they are the cheapest.
    \item He orders the remaining $(1-\beta-\gamma)$ trackers by the anonymity set size. 
    \item Adversary shuts 1-touchers starting from the least anonymous ones.
\end{itemize}

\begin{lemma}
Let fraction $\alpha$ of proposers be honest and alive. Then the total number of 0-touchers before final filtering is at most $$
W(\alpha) =\frac{N+20\sqrt{N}}{\alpha K}
$$
\end{lemma}
\begin{proof} Let us find how many trackers can evade the fraction $\alpha$ of stirs and thus become 0-touchers.

Consider any $W>K$ trackers and a single stir. The probability for all the trackers to miss the stir is $p=(1-\frac{K}{N})^W\approx e^{-\frac{WK}{N}}$. Consider  stirs $\mathcal{S}_j=\{S[1,j],S[2,j],\ldots,S[s,j]\}$ i.e. those  that are $j$-th in their round. The number $\nu_j$ of stirs in $S$ not touching any of those $W$ trackers is a random variable, which is the sum of $s$ independent Bernoulli variables with mean $p$, and so has Binomial distribution with parameters $(s,p)$. Note that random variables $\nu_j$  have negative covariance, so the total number $\nu$ of stirs not touching any of $W$ trackers  can be upper bounded by the variable with distribution $Bin(N/K\cdot s= N/2,p)$. The latter distribution can be approximated by normal one with parameters $(\mu=Np/2,\sigma^2=Np(1-p)/2)$.  With probability $e^{-80}$ we have that  the value of $\nu$ is at most $$
X(W)=\mu+12\sigma = Ne^{-\frac{WK}{N}}/2+12e^{-\frac{WK}{2N}}\sqrt{N/2}\approx \frac{N^2/2+10N\sqrt{N}}{WK}
$$. We thus assume that at most $2^{-128}$ such sets of $W$ trackers miss more than $X$ stirs, and they would be infeasible to find. Thus the total fraction of stirs that can be evaded is $X/(N/2) = \frac{N+20\sqrt{N}}{WK}$, and so is  the maximum fraction of honest proposers that miss $W$ trackers.
\end{proof}
For $N = K^2 = 2^{16}$ we have $W(
\alpha) = \frac{148}{\alpha}$.

Now note the following
\begin{itemize}
    \item The anonymity set of each tracker increases with each honest stir it undergoes. 
    \item Denote the fraction of 1-touchers after $r$ rounds by $F_1(r)$. Assuming uniform distribution of honest and online proposers over rounds, we have
    $$
    F_1(r)\approx 1-(1-\alpha)^r.
    $$
    This approximation is good enough for the anonymity set estimate.
    \item Thanks to the uniformity of dispersion, we have each stir of round $r+1$ taking the same fraction of 1- and 0-touchers. Therefore out of $(1-\beta)K$ benign trackers we have $F_1(r)(1-\beta)K$ 1-touchers each with anonymity set at least $(1-\beta)K$, and the anonymity set of 1-touchers last touched at round $r+1$ is at least
    $$
    A_1(r+1) = (1-(1-\alpha)^r)(1-\beta)^2K^2 
    $$
\end{itemize}


\begin{lemma} Suppose that the attacker shuts in day 2 down all nodes that were last touched in round $r$ or earlier of day 1. Then the cost is
\begin{equation}
   C(r)\geq \sum_{k=1}^r A_1(k)(F_1(k)-F_1(k-1))N \approx\sum_{k=1}^r (1-(\beta+\gamma+\delta)^{k-1})((1-\alpha)^{k-1}-(1-\alpha)^k)(1-\beta)^2K^2N.
\end{equation}

\end{lemma}

\begin{theorem} Let the adversary
\begin{itemize}
    \item control $\beta N$ proposers in period $i$ (i.e. she knows the shuffles of those).
    \item be able to shut down arbitrary $\delta N$ proposers in period $i$. 
\end{itemize} 
Let the fraction $\gamma$ of honest proposers of each day go offline. 
Then the fraction of honest proposers of each day is $\alpha = 1- \beta-\gamma-\delta$ and the cost of the attack is $C(r_0)$
 where $$
r_0= \log_{1-\alpha}(1-\delta + \frac{W(\alpha)}{2N})
$$

\end{theorem}
\begin{proof} In day 1 the fraction of honest and alive proposers who stir is $\alpha = 1- \beta-\gamma-\delta$. By Lemma the maximum number of trackers that become 0-touchers is $W(\alpha)$, of which 1/2 goes to Day 2. Therefore, out of $\delta N$ attacked trackers in Day 2, at least $\delta N - W(\alpha)/2$ are 1-touchers. The attacker then shuts down the 1-touchers that were touched at round $r_0$ at latest such that
$$
F_1(r_0)\geq \delta - \frac{W(\alpha)}{2N}
$$
which is equivalent to $$
r_0\geq \log_{1-\alpha}(1-\delta + \frac{W(\alpha)}{2N})
$$
The cost is then $C(r_0)$.
\end{proof}

\end{document}